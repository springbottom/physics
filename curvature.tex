\documentclass[12pt]{article}
\usepackage[margin=1in]{geometry} 
\usepackage{amsmath,amsthm,amssymb,amsfonts}
  
\newenvironment{problem}[2][Exercise]{\begin{trivlist}
\item[\hskip \labelsep {\bfseries #1}\hskip \labelsep {\bfseries #2.}]}{\end{trivlist}}
 %If you want to title your bold things something different just make another thing exactly like this but replace "problem" with the name of the thing you want, like theorem or lemma or whatever
    
\renewenvironment{proof}{{\bfseries Solution:}}{}
%%%%%%%%%%%%%%%%%%%%%%%%%%%%%%%%%%%%%%%%%%% IMPORTANT: THE ABOVE LINE MAKES "PROOFS" INTO SOLUTIONS IN CASE U R PHYSICS OR SOMETHING

\begin{document}
      
\title{Curvature of space as a time-independent perturbation}
\author{Joshua Lin\\Enrichment activity for Physics 137B, Fall 2017, Professor A.Charman}

\maketitle
       
%%%%%%%%%%%%%%%%%%%%% 
%\begin{mdframed}
%\begin{problem}{0}

%\end{problem}
%\end{mdframed}
        
%\begin{proof}

%\end{proof}
%%%%%%%%%%%%%%%%%%%%%

\section{Overview}

In our quantum physics lectures, both 137A and B, we were introduced to the Hamiltonian operator, which takes the form:
$$H = - \frac{\hbar^2}{2m} \nabla^2 + V$$
where $V$ is our potential. In class, we have only dealt with the case where we are interested in solutions to Schrodinger's equation:
$$i \hbar \frac{d}{dt} | \psi \rangle = H | \psi \rangle$$
in flat space (usually one, two or three dimensional Euclidean space, and occassionally compact spaces such as a closed interval (infinite square well) or a circle (question in homework on calculating energy levels of a fidget spinner)). In flat space, we can always find a cartesian set of coordinates to describe the space, at least locally, which leads to a laplacian of the form:
$$\nabla^2_{\text{flat}} = \partial_i \partial_i$$
using the Einstein summation notation, and taking $i$ to range over a cartesian basis. But what if we wanted to solve Schrodinger's equation on a more general manifold? If this manifold has geometry described by a metric $g_{ij}$, then our laplacian is generalised to the Laplace-Beltrami operator, and is given by:
$$\nabla^2_{\text{curved}} = \frac{1}{\sqrt{|g|}} \partial_i (\sqrt{|g|} g^{ij} \partial_j)$$
(Note that in the case of Euclidean space, our metric $g_{ij}$ is the delta function, and our operator reduces to $\partial_i \partial_i$ as expected). Now, assuming that the deviation of our new $\nabla^2$ operator from the old $\nabla^2$ operator in flat space is small in some sense, then it might be interesting to consider this as a time independent perturbation problem, namely:
$$H = - \frac{\hbar^2}{2m} \nabla^2_{\text{curved}} + V = \bigg(- \frac{\hbar^2}{2m} \nabla^2_{\text{flat}} + V\bigg) - \frac{\hbar^2}{2m} (\nabla^2_{\text{curved}} - \nabla^2_{\text{flat}})$$
where our perturbation is given by:
$$H_1 = -\frac{\hbar^2}{2m} (\nabla^2_{\text{curved}} - \nabla^2_{\text{flat}})$$

\section{Hyperbolic Geometry}
One of the most studied non-euclidean geometries is that of Hyperbolic Space, whose defining feature is a constant negative sectional curvature. As a model for the hyperbolic plane, consider a two dimensional space described by "geodesic-polar" coordinates $r, \theta$, which are the analogue to polar coordinates in the Euclidean plane. The metric will be given by:
$$g_{rr} = 1, g_{\theta \theta} = S_k^2(r), g_{r \theta} = g_{\theta r} = 0$$
where
$$S_k(r) = \frac{1}{\sqrt{-k}} \sinh(\sqrt{-k} r)$$
and $k$ is a negative real number. Notice firstly that in the limit as $k \to 0^-$, we have $g_{\theta \theta} \to r^2$, which is the usual metric for polar coordinates of the Euclidean plane. Now, we can verify that the curvature of this surface is constant everywhere by standard computations:
$$\Gamma^d_{ab} = \frac{1}{2} g^{cd}(\partial_b g_{ca} + \partial_a g_{cb} - \partial_c g_{ab})$$
To find the Christoffel symbols, which come out to be:
$$\Gamma^{r}_{\theta \theta} = - \frac{1}{2} \partial_r S_k^2; \Gamma^\theta_{r \theta} = \Gamma^\theta_{\theta r} = \frac{1}{2 S_k^2} \partial_r S_k^2$$
with all other components as zero. Now, we can compute the curvature tensor:
$$R^l_{ijk} = \frac{\partial}{\partial x^j} \Gamma^{l}_{ik} - \frac{\partial}{\partial x^k} \Gamma^l_{ij} + \Gamma^l_{js}\Gamma^s_{ik} - \Gamma^l_{ks} \Gamma^s_{ij}$$
which come out to be:
$$R^r_{\theta r \theta} = -R^r_{\theta \theta r} = \partial_r \Gamma^r_{\theta \theta} - \Gamma^r_{\theta \theta} \Gamma^\theta_{r \theta}; R^{\theta}_{r r \theta} = - R^\theta_{r \theta r} = \partial_r \Gamma^\theta_{r \theta} + \Gamma^{\theta}_{r \theta}\Gamma^{\theta}_{r \theta}$$
And finally, we can find the sectional curvature (which coincides with the Gaussian curvature; since it is a two-manifold):
$$K = \frac{g_{ab}R^{a}_{cde}u^ev^dv^cu^b}{g_{a'b'}u^{a'}u^{b'}g_{c'd'}v^{c'}v^{d'} - (g_{e'f'}u^{e'}v^{f'})^2}$$
where $u$, $v$ are linearly independent vectors in the tangent space (note the curvature is invariant under different choices of $u$ and $v$, as it should be). Choosing $u$ to be the unit vector in the $\theta$ direction and $v$ to be the unit vector in the $r$ direction, we find:
$$K = k$$
everywhere, which is exactly what we expect. Now that we have verified that our model is indeed the hyperbolic plane (with constant curvature $k$), we can find the laplacian:
$$\nabla^2_{\text{curved}} = \frac{1}{\sqrt{|g|}} \partial_i (\sqrt{|g|} g^{ij} \partial_j)$$
$$= \bigg(\frac{1}{r} \partial_r + \partial_r^2 + \frac{1}{r^2} \partial_{\theta}^2 \bigg) + \bigg(\frac{1}{S_k^2} - \frac{1}{r^2}\bigg)\partial_\theta^2$$
where the first part in the above equation is simply the laplacian if our polar coordinates described flat space. So our perturbation in the hyperbolic plane has form:
$$H_1 = - \frac{\hbar^2}{2m} \bigg( \frac{1}{S_k^2} - \frac{1}{r^2} \bigg) \partial_\theta^2$$


\section{Application to Quantum Harmonic Oscillator in Hyperbolic Plane}
Now, we can consider the Quantum Harmonic Oscillator in the plane, which has well-known solution in the case where the plane is flat. The wavefunctions of the one-dimensional Harmonic Oscillator are given by: 
$$\psi_n(x) = A_n H_n \bigg( \frac{x}{\sigma_0}\bigg) e^{-x^2/2\sigma_0^2};\qquad A_n := \frac{1}{\sqrt{2^n n!}} \frac{1}{(\pi \sigma_0^2)^{1/4}};\qquad \sigma_0 := (\hbar/m\omega)^{1/2} $$
where $H_n$ are the physicist's Hermite polynomials, with corresponding energy levels:
$$E_n = \bigg( n + \frac{1}{2} \bigg) \hbar \omega$$
Our solution to the two dimensional case, being separable, is given by:
$$\psi_{nm}(x,y) = \psi_n(x) \psi_m(y); \qquad E_{nm} = (n + m + 1) \hbar \omega$$
Now, suppose our curvature $k$ is small in magnitude. Then, if we are only interested in first order effects, we can do first order perturbation theory; whilst taylor expanding our perturbation in $k$ to find:
$$H_1 \approx -\frac{\hbar^2}{2m} \frac{k}{3} \partial_{\theta}^2$$
to first order in $k$. Applying first order perturbation theory, we find the change in energy levels due to perturbation:
$$\langle n,m | H_1 | n,m \rangle = -\frac{\hbar^3 k }{6m^2 \omega} A_n^2 A_m^2 \int du \int dv e^{-(u^2 + v^2)} \bigg(H_n( u ) H_m (v )\bigg) \partial_\theta^2 \bigg(H_n( u) H_m (v)\bigg) $$
where we make the coordinate transformation $u = x/\sigma_0$ and $v = y/\sigma_0$ (Note that $\theta = \text{atan} (v/u)$). This may seem slightly problematic for two reasons. Conceptually; in class we've always understood perturbations to be perturbations in the potential energy, which only depend on spatial degrees (not their derivatives). The 'proof' of the validity of time independent perturbation theory (matching coefficients in orders of the perturbation) doesn't require that the perturbation be only in terms of spatial degrees however, so we are fine in that regard. Computationally; this might look like an ugly integral, but actually it works just fine:
$$\partial_\theta^2 \bigg( H_n(u) H_m(v) \bigg) = (\partial_\theta^2 H_n) H_m + 2(\partial_\theta H_n) (\partial_\theta H_m) + H_n (\partial_\theta^2 H_m)$$
$$ = (y^2 \partial_u^2 -x \partial_u)H_n(u) H_m(v) -2uv (\partial_u H_n(u))(\partial_v H_m(v)) + H_n(u)(u^2 \partial_v^2 - v \partial_v)H_m(v)$$
Notice that we have:
$$\partial_z H_n(z) = 2z H_n(z) - H_{n+1}(z)$$
Using this formula, the above reduces to:
$$\partial_\theta^2 \bigg( H_n(u) H_m(v) \bigg) = u^2 H_n(u) H_{m+2}(v) -2uvH_{n+1}(u)H_{m+1}(v) + v^2 H_{n+2}(u) H_m(v)$$
$$-vH_n(u)H_{m+1}(v) -u H_{n+1}(u) H_m (v)$$
Plugging this back into our original integral, the $H_n H_{m+2}$ and $H_{n+2}H_m$ terms vanish by orthogonality of the eigenstates of the one-dimensional harmonic oscillator (the magic of mathematics!) so we end with:
$$\langle n,m | H_1 | n,m \rangle =  \frac{\hbar^3 k }{6m^2 \omega} A_n^2 A_m^2 \int du \int dv e^{-(u^2 + v^2)} \bigg[ 2uv H_n(u) H_m(v) H_{n+1}(u) H_{m+1}(v)$$
$$-v H_n(u)^2 H_m(v) H_{m+1}(v) -u H_n(u) H_{n+1}(u) H_m(v)^2 \bigg]$$
Expressing this in other terms, we have:
$$\langle n,m | H_1 | n,m \rangle =  \frac{\hbar^2 k}{6 m} \bigg[ 2 \frac{A_n}{A_{n+1}} \frac{A_m}{A_{m+1}} \langle n | \bigg( \frac{\hat{a} + \hat{a^{\dagger}}}{\sqrt{2}} \bigg) | n+1 \rangle \langle m | \bigg( \frac{\hat{a} + \hat{a^{\dagger}}}{\sqrt{2}} \bigg) | m+1 \rangle$$
$$- \frac{A_n}{A_{n+1}} \langle n | \bigg( \frac{\hat{a} + \hat{a^{\dagger}}}{\sqrt{2}} \bigg) | n+1 \rangle - \frac{A_m}{A_{m+1}} \langle m | \bigg( \frac{\hat{a} + \hat{a^{\dagger}}}{\sqrt{2}} \bigg) | m+1 \rangle \bigg]  $$
where $\hat{a}$ and $\hat{a^{\dagger}}$ are our annihilation and creation operators respectively. Our final formula is:
$$\langle n,m | H_1 | n,m \rangle =  \frac{\hbar^2 k}{6 m} (2nm+n+m)$$

\section{Behaviour of Solution}

Recall that the curvature $k$ is defined to be negative for the hyperbolic plane, so the perturbation due to negative curvature lowers the energies of all the eigenstates. Also, due to the $nm$ factor in the perturbation energy, the energy levels increase slower and slower (as $n+m$ increases) to a fixed maximum energy, before decreasing. All of these are features of the exact solution of the quantum harmonic oscillator in the hyperbolic plane (https://arxiv.org/pdf/0709.0399.pdf), and it turns out that there are only a finite number of eigenstates (only for the $n+m$ that yield energies below the maximum energy). Our perturbative method didn't yield the exact energies derived in the paper, but the fact that it gave the same characteristics as the exact solution might be seen as a positive.  



\end{document}


